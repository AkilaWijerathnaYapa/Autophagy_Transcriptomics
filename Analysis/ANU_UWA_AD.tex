\documentclass[]{article}
\usepackage{lmodern}
\usepackage{amssymb,amsmath}
\usepackage{ifxetex,ifluatex}
\usepackage{fixltx2e} % provides \textsubscript
\ifnum 0\ifxetex 1\fi\ifluatex 1\fi=0 % if pdftex
  \usepackage[T1]{fontenc}
  \usepackage[utf8]{inputenc}
\else % if luatex or xelatex
  \ifxetex
    \usepackage{mathspec}
  \else
    \usepackage{fontspec}
  \fi
  \defaultfontfeatures{Ligatures=TeX,Scale=MatchLowercase}
\fi
% use upquote if available, for straight quotes in verbatim environments
\IfFileExists{upquote.sty}{\usepackage{upquote}}{}
% use microtype if available
\IfFileExists{microtype.sty}{%
\usepackage{microtype}
\UseMicrotypeSet[protrusion]{basicmath} % disable protrusion for tt fonts
}{}
\usepackage[margin=1in]{geometry}
\usepackage{hyperref}
\hypersetup{unicode=true,
            pdftitle={Photomorphogenesis\_ATG},
            pdfauthor={Akila \& Diep},
            pdfborder={0 0 0},
            breaklinks=true}
\urlstyle{same}  % don't use monospace font for urls
\usepackage{color}
\usepackage{fancyvrb}
\newcommand{\VerbBar}{|}
\newcommand{\VERB}{\Verb[commandchars=\\\{\}]}
\DefineVerbatimEnvironment{Highlighting}{Verbatim}{commandchars=\\\{\}}
% Add ',fontsize=\small' for more characters per line
\usepackage{framed}
\definecolor{shadecolor}{RGB}{248,248,248}
\newenvironment{Shaded}{\begin{snugshade}}{\end{snugshade}}
\newcommand{\AlertTok}[1]{\textcolor[rgb]{0.94,0.16,0.16}{#1}}
\newcommand{\AnnotationTok}[1]{\textcolor[rgb]{0.56,0.35,0.01}{\textbf{\textit{#1}}}}
\newcommand{\AttributeTok}[1]{\textcolor[rgb]{0.77,0.63,0.00}{#1}}
\newcommand{\BaseNTok}[1]{\textcolor[rgb]{0.00,0.00,0.81}{#1}}
\newcommand{\BuiltInTok}[1]{#1}
\newcommand{\CharTok}[1]{\textcolor[rgb]{0.31,0.60,0.02}{#1}}
\newcommand{\CommentTok}[1]{\textcolor[rgb]{0.56,0.35,0.01}{\textit{#1}}}
\newcommand{\CommentVarTok}[1]{\textcolor[rgb]{0.56,0.35,0.01}{\textbf{\textit{#1}}}}
\newcommand{\ConstantTok}[1]{\textcolor[rgb]{0.00,0.00,0.00}{#1}}
\newcommand{\ControlFlowTok}[1]{\textcolor[rgb]{0.13,0.29,0.53}{\textbf{#1}}}
\newcommand{\DataTypeTok}[1]{\textcolor[rgb]{0.13,0.29,0.53}{#1}}
\newcommand{\DecValTok}[1]{\textcolor[rgb]{0.00,0.00,0.81}{#1}}
\newcommand{\DocumentationTok}[1]{\textcolor[rgb]{0.56,0.35,0.01}{\textbf{\textit{#1}}}}
\newcommand{\ErrorTok}[1]{\textcolor[rgb]{0.64,0.00,0.00}{\textbf{#1}}}
\newcommand{\ExtensionTok}[1]{#1}
\newcommand{\FloatTok}[1]{\textcolor[rgb]{0.00,0.00,0.81}{#1}}
\newcommand{\FunctionTok}[1]{\textcolor[rgb]{0.00,0.00,0.00}{#1}}
\newcommand{\ImportTok}[1]{#1}
\newcommand{\InformationTok}[1]{\textcolor[rgb]{0.56,0.35,0.01}{\textbf{\textit{#1}}}}
\newcommand{\KeywordTok}[1]{\textcolor[rgb]{0.13,0.29,0.53}{\textbf{#1}}}
\newcommand{\NormalTok}[1]{#1}
\newcommand{\OperatorTok}[1]{\textcolor[rgb]{0.81,0.36,0.00}{\textbf{#1}}}
\newcommand{\OtherTok}[1]{\textcolor[rgb]{0.56,0.35,0.01}{#1}}
\newcommand{\PreprocessorTok}[1]{\textcolor[rgb]{0.56,0.35,0.01}{\textit{#1}}}
\newcommand{\RegionMarkerTok}[1]{#1}
\newcommand{\SpecialCharTok}[1]{\textcolor[rgb]{0.00,0.00,0.00}{#1}}
\newcommand{\SpecialStringTok}[1]{\textcolor[rgb]{0.31,0.60,0.02}{#1}}
\newcommand{\StringTok}[1]{\textcolor[rgb]{0.31,0.60,0.02}{#1}}
\newcommand{\VariableTok}[1]{\textcolor[rgb]{0.00,0.00,0.00}{#1}}
\newcommand{\VerbatimStringTok}[1]{\textcolor[rgb]{0.31,0.60,0.02}{#1}}
\newcommand{\WarningTok}[1]{\textcolor[rgb]{0.56,0.35,0.01}{\textbf{\textit{#1}}}}
\usepackage{graphicx,grffile}
\makeatletter
\def\maxwidth{\ifdim\Gin@nat@width>\linewidth\linewidth\else\Gin@nat@width\fi}
\def\maxheight{\ifdim\Gin@nat@height>\textheight\textheight\else\Gin@nat@height\fi}
\makeatother
% Scale images if necessary, so that they will not overflow the page
% margins by default, and it is still possible to overwrite the defaults
% using explicit options in \includegraphics[width, height, ...]{}
\setkeys{Gin}{width=\maxwidth,height=\maxheight,keepaspectratio}
\IfFileExists{parskip.sty}{%
\usepackage{parskip}
}{% else
\setlength{\parindent}{0pt}
\setlength{\parskip}{6pt plus 2pt minus 1pt}
}
\setlength{\emergencystretch}{3em}  % prevent overfull lines
\providecommand{\tightlist}{%
  \setlength{\itemsep}{0pt}\setlength{\parskip}{0pt}}
\setcounter{secnumdepth}{0}
% Redefines (sub)paragraphs to behave more like sections
\ifx\paragraph\undefined\else
\let\oldparagraph\paragraph
\renewcommand{\paragraph}[1]{\oldparagraph{#1}\mbox{}}
\fi
\ifx\subparagraph\undefined\else
\let\oldsubparagraph\subparagraph
\renewcommand{\subparagraph}[1]{\oldsubparagraph{#1}\mbox{}}
\fi

%%% Use protect on footnotes to avoid problems with footnotes in titles
\let\rmarkdownfootnote\footnote%
\def\footnote{\protect\rmarkdownfootnote}

%%% Change title format to be more compact
\usepackage{titling}

% Create subtitle command for use in maketitle
\providecommand{\subtitle}[1]{
  \posttitle{
    \begin{center}\large#1\end{center}
    }
}

\setlength{\droptitle}{-2em}

  \title{Photomorphogenesis\_ATG}
    \pretitle{\vspace{\droptitle}\centering\huge}
  \posttitle{\par}
    \author{Akila \& Diep}
    \preauthor{\centering\large\emph}
  \postauthor{\par}
      \predate{\centering\large\emph}
  \postdate{\par}
    \date{29/09/2019}


\begin{document}
\maketitle

\hypertarget{autophagy-deetiolation-rnaseq-analysis}{%
\section{Autophagy Deetiolation RNASeq
Analysis}\label{autophagy-deetiolation-rnaseq-analysis}}

\hypertarget{peb-internode-colloboration-grant}{%
\section{PEB Internode Colloboration
Grant}\label{peb-internode-colloboration-grant}}

\hypertarget{diep-akila}{%
\section{Diep \& Akila}\label{diep-akila}}

\begin{Shaded}
\begin{Highlighting}[]
\CommentTok{# Load Packages }

\KeywordTok{library}\NormalTok{(ThreeDRNAseq)}
\KeywordTok{library}\NormalTok{(limma)}
\KeywordTok{library}\NormalTok{(edgeR)}
\KeywordTok{library}\NormalTok{(tximport)}
\KeywordTok{library}\NormalTok{(tidyverse)}
\end{Highlighting}
\end{Shaded}

\begin{verbatim}
## -- Attaching packages ------------------------------------------------------------------ tidyverse 1.2.1 --
\end{verbatim}

\begin{verbatim}
## v ggplot2 3.2.0     v purrr   0.3.2
## v tibble  2.1.3     v dplyr   0.8.3
## v tidyr   0.8.3     v stringr 1.4.0
## v readr   1.3.1     v forcats 0.4.0
\end{verbatim}

\begin{verbatim}
## -- Conflicts --------------------------------------------------------------------- tidyverse_conflicts() --
## x dplyr::filter() masks stats::filter()
## x dplyr::lag()    masks stats::lag()
\end{verbatim}

\begin{Shaded}
\begin{Highlighting}[]
\KeywordTok{library}\NormalTok{(tximport)}
\KeywordTok{library}\NormalTok{(RUVSeq)}
\end{Highlighting}
\end{Shaded}

\begin{verbatim}
## Loading required package: Biobase
\end{verbatim}

\begin{verbatim}
## Loading required package: BiocGenerics
\end{verbatim}

\begin{verbatim}
## Loading required package: parallel
\end{verbatim}

\begin{verbatim}
## 
## Attaching package: 'BiocGenerics'
\end{verbatim}

\begin{verbatim}
## The following objects are masked from 'package:parallel':
## 
##     clusterApply, clusterApplyLB, clusterCall, clusterEvalQ,
##     clusterExport, clusterMap, parApply, parCapply, parLapply,
##     parLapplyLB, parRapply, parSapply, parSapplyLB
\end{verbatim}

\begin{verbatim}
## The following objects are masked from 'package:dplyr':
## 
##     combine, intersect, setdiff, union
\end{verbatim}

\begin{verbatim}
## The following object is masked from 'package:limma':
## 
##     plotMA
\end{verbatim}

\begin{verbatim}
## The following objects are masked from 'package:stats':
## 
##     IQR, mad, sd, var, xtabs
\end{verbatim}

\begin{verbatim}
## The following objects are masked from 'package:base':
## 
##     anyDuplicated, append, as.data.frame, basename, cbind,
##     colnames, dirname, do.call, duplicated, eval, evalq, Filter,
##     Find, get, grep, grepl, intersect, is.unsorted, lapply, Map,
##     mapply, match, mget, order, paste, pmax, pmax.int, pmin,
##     pmin.int, Position, rank, rbind, Reduce, rownames, sapply,
##     setdiff, sort, table, tapply, union, unique, unsplit, which,
##     which.max, which.min
\end{verbatim}

\begin{verbatim}
## Welcome to Bioconductor
## 
##     Vignettes contain introductory material; view with
##     'browseVignettes()'. To cite Bioconductor, see
##     'citation("Biobase")', and for packages 'citation("pkgname")'.
\end{verbatim}

\begin{verbatim}
## Loading required package: EDASeq
\end{verbatim}

\begin{verbatim}
## Loading required package: ShortRead
\end{verbatim}

\begin{verbatim}
## Loading required package: BiocParallel
\end{verbatim}

\begin{verbatim}
## Loading required package: Biostrings
\end{verbatim}

\begin{verbatim}
## Loading required package: S4Vectors
\end{verbatim}

\begin{verbatim}
## Loading required package: stats4
\end{verbatim}

\begin{verbatim}
## 
## Attaching package: 'S4Vectors'
\end{verbatim}

\begin{verbatim}
## The following objects are masked from 'package:dplyr':
## 
##     first, rename
\end{verbatim}

\begin{verbatim}
## The following object is masked from 'package:tidyr':
## 
##     expand
\end{verbatim}

\begin{verbatim}
## The following object is masked from 'package:base':
## 
##     expand.grid
\end{verbatim}

\begin{verbatim}
## Loading required package: IRanges
\end{verbatim}

\begin{verbatim}
## 
## Attaching package: 'IRanges'
\end{verbatim}

\begin{verbatim}
## The following objects are masked from 'package:dplyr':
## 
##     collapse, desc, slice
\end{verbatim}

\begin{verbatim}
## The following object is masked from 'package:purrr':
## 
##     reduce
\end{verbatim}

\begin{verbatim}
## The following object is masked from 'package:grDevices':
## 
##     windows
\end{verbatim}

\begin{verbatim}
## Loading required package: XVector
\end{verbatim}

\begin{verbatim}
## 
## Attaching package: 'XVector'
\end{verbatim}

\begin{verbatim}
## The following object is masked from 'package:purrr':
## 
##     compact
\end{verbatim}

\begin{verbatim}
## 
## Attaching package: 'Biostrings'
\end{verbatim}

\begin{verbatim}
## The following object is masked from 'package:base':
## 
##     strsplit
\end{verbatim}

\begin{verbatim}
## Loading required package: Rsamtools
\end{verbatim}

\begin{verbatim}
## Loading required package: GenomeInfoDb
\end{verbatim}

\begin{verbatim}
## Loading required package: GenomicRanges
\end{verbatim}

\begin{verbatim}
## Loading required package: GenomicAlignments
\end{verbatim}

\begin{verbatim}
## Loading required package: SummarizedExperiment
\end{verbatim}

\begin{verbatim}
## Loading required package: DelayedArray
\end{verbatim}

\begin{verbatim}
## Loading required package: matrixStats
\end{verbatim}

\begin{verbatim}
## 
## Attaching package: 'matrixStats'
\end{verbatim}

\begin{verbatim}
## The following objects are masked from 'package:Biobase':
## 
##     anyMissing, rowMedians
\end{verbatim}

\begin{verbatim}
## The following object is masked from 'package:dplyr':
## 
##     count
\end{verbatim}

\begin{verbatim}
## 
## Attaching package: 'DelayedArray'
\end{verbatim}

\begin{verbatim}
## The following objects are masked from 'package:matrixStats':
## 
##     colMaxs, colMins, colRanges, rowMaxs, rowMins, rowRanges
\end{verbatim}

\begin{verbatim}
## The following object is masked from 'package:Biostrings':
## 
##     type
\end{verbatim}

\begin{verbatim}
## The following object is masked from 'package:purrr':
## 
##     simplify
\end{verbatim}

\begin{verbatim}
## The following objects are masked from 'package:base':
## 
##     aperm, apply, rowsum
\end{verbatim}

\begin{verbatim}
## 
## Attaching package: 'GenomicAlignments'
\end{verbatim}

\begin{verbatim}
## The following object is masked from 'package:dplyr':
## 
##     last
\end{verbatim}

\begin{verbatim}
## 
## Attaching package: 'ShortRead'
\end{verbatim}

\begin{verbatim}
## The following object is masked from 'package:dplyr':
## 
##     id
\end{verbatim}

\begin{verbatim}
## The following object is masked from 'package:purrr':
## 
##     compose
\end{verbatim}

\begin{verbatim}
## The following object is masked from 'package:tibble':
## 
##     view
\end{verbatim}

\begin{verbatim}
## Registered S3 method overwritten by 'R.oo':
##   method        from       
##   throw.default R.methodsS3
\end{verbatim}

\begin{Shaded}
\begin{Highlighting}[]
\KeywordTok{library}\NormalTok{(eulerr)}
\KeywordTok{library}\NormalTok{(gridExtra)}
\end{Highlighting}
\end{Shaded}

\begin{verbatim}
## 
## Attaching package: 'gridExtra'
\end{verbatim}

\begin{verbatim}
## The following object is masked from 'package:Biobase':
## 
##     combine
\end{verbatim}

\begin{verbatim}
## The following object is masked from 'package:BiocGenerics':
## 
##     combine
\end{verbatim}

\begin{verbatim}
## The following object is masked from 'package:dplyr':
## 
##     combine
\end{verbatim}

\begin{Shaded}
\begin{Highlighting}[]
\KeywordTok{library}\NormalTok{(grid)}
\KeywordTok{library}\NormalTok{(ComplexHeatmap)}
\end{Highlighting}
\end{Shaded}

\begin{verbatim}
## ========================================
## ComplexHeatmap version 2.0.0
## Bioconductor page: http://bioconductor.org/packages/ComplexHeatmap/
## Github page: https://github.com/jokergoo/ComplexHeatmap
## Documentation: http://jokergoo.github.io/ComplexHeatmap-reference
## 
## If you use it in published research, please cite:
## Gu, Z. Complex heatmaps reveal patterns and correlations in multidimensional 
##   genomic data. Bioinformatics 2016.
## ========================================
\end{verbatim}

\begin{Shaded}
\begin{Highlighting}[]
\KeywordTok{library}\NormalTok{(ggrepel)}
\end{Highlighting}
\end{Shaded}

\hypertarget{arrange-sample-information}{%
\subsection{arrange sample
information}\label{arrange-sample-information}}

\begin{Shaded}
\begin{Highlighting}[]
\NormalTok{metadata <-}\StringTok{ }\KeywordTok{tibble}\NormalTok{(}\DataTypeTok{samples =} \KeywordTok{dir}\NormalTok{(}\StringTok{"tsv_files"}\NormalTok{),}
            \DataTypeTok{condition =} \KeywordTok{sapply}\NormalTok{(}\KeywordTok{strsplit}\NormalTok{(samples, }\StringTok{"_"}\NormalTok{), }\ControlFlowTok{function}\NormalTok{(l) l[}\DecValTok{1}\NormalTok{]))}

\NormalTok{metadata <-}\StringTok{ }\KeywordTok{tibble}\NormalTok{(}\DataTypeTok{samples =} \KeywordTok{dir}\NormalTok{(}\StringTok{"tsv_files"}\NormalTok{),}
            \DataTypeTok{label =} \KeywordTok{str_sub}\NormalTok{(metadata}\OperatorTok{$}\NormalTok{samples, }\DataTypeTok{end =} \DecValTok{-5}\NormalTok{),}
            \DataTypeTok{genotype =} \KeywordTok{sapply}\NormalTok{(}\KeywordTok{strsplit}\NormalTok{(samples, }\StringTok{"_"}\NormalTok{), }\ControlFlowTok{function}\NormalTok{(l) l[}\DecValTok{1}\NormalTok{]),}
            \DataTypeTok{time =} \KeywordTok{sapply}\NormalTok{(}\KeywordTok{strsplit}\NormalTok{(samples, }\StringTok{"_"}\NormalTok{), }\ControlFlowTok{function}\NormalTok{(l) l[}\DecValTok{2}\NormalTok{]),}
            \DataTypeTok{treat =} \KeywordTok{paste0}\NormalTok{(genotype,}\StringTok{'_'}\NormalTok{,time))}


\NormalTok{files <-}\StringTok{ }\KeywordTok{file.path}\NormalTok{(}\StringTok{"tsv_files"}\NormalTok{, metadata}\OperatorTok{$}\NormalTok{samples)}

\KeywordTok{names}\NormalTok{(files) <-}\StringTok{ }\NormalTok{metadata}\OperatorTok{$}\NormalTok{samples}

\KeywordTok{all}\NormalTok{(}\KeywordTok{file.exists}\NormalTok{(files))}
\end{Highlighting}
\end{Shaded}

\begin{verbatim}
## [1] TRUE
\end{verbatim}

\hypertarget{annotation-function}{%
\subsection{annotation function}\label{annotation-function}}

\begin{Shaded}
\begin{Highlighting}[]
\NormalTok{getAttributeField <-}\StringTok{ }\ControlFlowTok{function}\NormalTok{ (x, field, }\DataTypeTok{attrsep =} \StringTok{";"}\NormalTok{) \{}
\NormalTok{                      s =}\StringTok{ }\KeywordTok{strsplit}\NormalTok{(x, }\DataTypeTok{split =}\NormalTok{ attrsep, }\DataTypeTok{fixed =} \OtherTok{TRUE}\NormalTok{)}
                      \KeywordTok{sapply}\NormalTok{(s, }\ControlFlowTok{function}\NormalTok{(atts) \{}
\NormalTok{                        a =}\StringTok{ }\KeywordTok{strsplit}\NormalTok{(atts, }\DataTypeTok{split =} \StringTok{"="}\NormalTok{, }\DataTypeTok{fixed =} \OtherTok{TRUE}\NormalTok{)}
\NormalTok{                        m =}\StringTok{ }\KeywordTok{match}\NormalTok{(field, }\KeywordTok{sapply}\NormalTok{(a, }\StringTok{"["}\NormalTok{, }\DecValTok{1}\NormalTok{))}
                        \ControlFlowTok{if}\NormalTok{ (}\OperatorTok{!}\KeywordTok{is.na}\NormalTok{(m)) \{}
\NormalTok{                          rv =}\StringTok{ }\NormalTok{a[[m]][}\DecValTok{2}\NormalTok{]}
\NormalTok{                        \}}
                        \ControlFlowTok{else}\NormalTok{ \{}
\NormalTok{                          rv =}\StringTok{ }\KeywordTok{as.character}\NormalTok{(}\OtherTok{NA}\NormalTok{)}
\NormalTok{                        \}}
                        \KeywordTok{return}\NormalTok{(rv)}
\NormalTok{                      \})}
\NormalTok{                    \}}
\end{Highlighting}
\end{Shaded}

\hypertarget{gene-info}{%
\subsection{gene info}\label{gene-info}}

\begin{Shaded}
\begin{Highlighting}[]
\NormalTok{athal_gene <-}\StringTok{ }\KeywordTok{read_delim}\NormalTok{(}\StringTok{"C:/Users/akila/OneDrive/Documents/R_Projects/Kallisto_OP/Analysis/Arabidopsis_thaliana.TAIR10.44.gff3"}\NormalTok{,}\DataTypeTok{skip=}\DecValTok{13}\NormalTok{,}\DataTypeTok{delim=}\StringTok{'}\CharTok{\textbackslash{}t}\StringTok{'}\NormalTok{,}\DataTypeTok{col_names =}\NormalTok{ F) }\OperatorTok
\StringTok{              }\KeywordTok{subset}\NormalTok{(X3 }\OperatorTok{==}\StringTok{ "gene"}\NormalTok{) }\OperatorTok
\StringTok{              }\KeywordTok{mutate}\NormalTok{(}\DataTypeTok{name =} \KeywordTok{getAttributeField}\NormalTok{(X9, }\StringTok{"Name"}\NormalTok{)) }\OperatorTok
\StringTok{              }\KeywordTok{mutate}\NormalTok{(}\DataTypeTok{gene =} \KeywordTok{getAttributeField}\NormalTok{(X9, }\StringTok{"ID"}\NormalTok{)) }\OperatorTok
\StringTok{              }\KeywordTok{mutate}\NormalTok{(}\DataTypeTok{gene =} \KeywordTok{sapply}\NormalTok{(}\KeywordTok{strsplit}\NormalTok{(gene,}\StringTok{":"}\NormalTok{), }\ControlFlowTok{function}\NormalTok{(l) l[}\DecValTok{2}\NormalTok{])) }\OperatorTok
\StringTok{              }\KeywordTok{mutate}\NormalTok{(}\DataTypeTok{description =} \KeywordTok{getAttributeField}\NormalTok{(X9, }\StringTok{"description"}\NormalTok{)) }\OperatorTok
\StringTok{              }\KeywordTok{mutate}\NormalTok{(}\DataTypeTok{description =} \KeywordTok{sapply}\NormalTok{(}\KeywordTok{strsplit}\NormalTok{(description, }\StringTok{" }\CharTok{\textbackslash{}\textbackslash{}}\StringTok{["}\NormalTok{), }\ControlFlowTok{function}\NormalTok{(l) l[}\DecValTok{1}\NormalTok{])) }\OperatorTok
\StringTok{              }\KeywordTok{select}\NormalTok{(name, gene, description)}
\end{Highlighting}
\end{Shaded}

\begin{verbatim}
## Parsed with column specification:
## cols(
##   X1 = col_character(),
##   X2 = col_character(),
##   X3 = col_character(),
##   X4 = col_double(),
##   X5 = col_double(),
##   X6 = col_character(),
##   X7 = col_character(),
##   X8 = col_character(),
##   X9 = col_character()
## )
\end{verbatim}

\begin{verbatim}
## Warning: 32840 parsing failures.
## row col  expected    actual                                                                                                    file
##   2  -- 9 columns 1 columns 'C:/Users/akila/OneDrive/Documents/R_Projects/Kallisto_OP/Analysis/Arabidopsis_thaliana.TAIR10.44.gff3'
##  19  -- 9 columns 1 columns 'C:/Users/akila/OneDrive/Documents/R_Projects/Kallisto_OP/Analysis/Arabidopsis_thaliana.TAIR10.44.gff3'
## 135  -- 9 columns 1 columns 'C:/Users/akila/OneDrive/Documents/R_Projects/Kallisto_OP/Analysis/Arabidopsis_thaliana.TAIR10.44.gff3'
## 139  -- 9 columns 1 columns 'C:/Users/akila/OneDrive/Documents/R_Projects/Kallisto_OP/Analysis/Arabidopsis_thaliana.TAIR10.44.gff3'
## 157  -- 9 columns 1 columns 'C:/Users/akila/OneDrive/Documents/R_Projects/Kallisto_OP/Analysis/Arabidopsis_thaliana.TAIR10.44.gff3'
## ... ... ......... ......... .......................................................................................................
## See problems(...) for more details.
\end{verbatim}

\begin{Shaded}
\begin{Highlighting}[]
\CommentTok{## transcript mapping info / This was called mapping (mapping.csv) in 3DRNASeq}
\NormalTok{tx2gene <-}\StringTok{ }\KeywordTok{read_delim}\NormalTok{(}\StringTok{"C:/Users/akila/OneDrive/Documents/R_Projects/Kallisto_OP/Analysis/AtRTD2_19April2016.gtf"}\NormalTok{, }\DataTypeTok{delim=}\StringTok{'}\CharTok{\textbackslash{}t}\StringTok{'}\NormalTok{, }\DataTypeTok{col_names =}\NormalTok{ F) }\OperatorTok
\StringTok{            }\KeywordTok{mutate}\NormalTok{(}\DataTypeTok{TXNAME =} \KeywordTok{sapply}\NormalTok{(}\KeywordTok{strsplit}\NormalTok{(X9, }\StringTok{'"'}\NormalTok{), }\ControlFlowTok{function}\NormalTok{(l) l[}\DecValTok{2}\NormalTok{])) }\OperatorTok
\StringTok{            }\KeywordTok{mutate}\NormalTok{(}\DataTypeTok{GENEID =} \KeywordTok{sapply}\NormalTok{(}\KeywordTok{strsplit}\NormalTok{(X9, }\StringTok{'"'}\NormalTok{), }\ControlFlowTok{function}\NormalTok{(l) l[}\DecValTok{4}\NormalTok{])) }\OperatorTok
\StringTok{            }\KeywordTok{select}\NormalTok{(TXNAME,GENEID) }\OperatorTok\StringTok{  }\NormalTok{unique}
\end{Highlighting}
\end{Shaded}

\begin{verbatim}
## Parsed with column specification:
## cols(
##   X1 = col_character(),
##   X2 = col_character(),
##   X3 = col_character(),
##   X4 = col_double(),
##   X5 = col_double(),
##   X6 = col_character(),
##   X7 = col_character(),
##   X8 = col_character(),
##   X9 = col_character()
## )
\end{verbatim}

\hypertarget{generate-gene-expression}{%
\subsection{generate gene expression}\label{generate-gene-expression}}

\begin{Shaded}
\begin{Highlighting}[]
\NormalTok{txi_gene <-}\StringTok{ }\KeywordTok{tximport}\NormalTok{(files, }\DataTypeTok{tx2gene =}\NormalTok{ tx2gene, }\StringTok{"kallisto"}\NormalTok{, }\DataTypeTok{countsFromAbundance =} \StringTok{"lengthScaledTPM"}\NormalTok{)}
\end{Highlighting}
\end{Shaded}

\begin{verbatim}
## Note: importing `abundance.h5` is typically faster than `abundance.tsv`
\end{verbatim}

\begin{verbatim}
## reading in files with read_tsv
\end{verbatim}

\begin{verbatim}
## 1 2 3 4 5 6 7 8 9 10 11 12 13 14 15 16 17 18 19 20 21 22 23 24 25 26 27 28 29 30 31 32 33 34 35 36 37 38 39 40 41 42 
## summarizing abundance
## summarizing counts
## summarizing length
\end{verbatim}

\hypertarget{generate-transcript-level-counts}{%
\subsection{generate transcript level
counts}\label{generate-transcript-level-counts}}

\begin{Shaded}
\begin{Highlighting}[]
\NormalTok{txi_trans <-}\StringTok{ }\KeywordTok{tximport}\NormalTok{(files, }\DataTypeTok{tx2gene =}\NormalTok{ tx2gene, }\StringTok{"kallisto"}\NormalTok{, }\DataTypeTok{txOut =}\NormalTok{ T, }\DataTypeTok{countsFromAbundance =} \StringTok{"dtuScaledTPM"}\NormalTok{)}
\end{Highlighting}
\end{Shaded}

\begin{verbatim}
## Note: importing `abundance.h5` is typically faster than `abundance.tsv`
\end{verbatim}

\begin{verbatim}
## reading in files with read_tsv
\end{verbatim}

\begin{verbatim}
## 1 2 3 4 5 6 7 8 9 10 11 12 13 14 15 16 17 18 19 20 21 22 23 24 25 26 27 28 29 30 31 32 33 34 35 36 37 38 39 40 41 42
\end{verbatim}

\#\#extract gene and transcript read counts

\hypertarget{take-genes_counts-through-data-pre-processing}{%
\subsection{take genes\_counts through data
pre-processing}\label{take-genes_counts-through-data-pre-processing}}

\begin{Shaded}
\begin{Highlighting}[]
\NormalTok{genes_counts <-}\StringTok{ }\NormalTok{txi_gene}\OperatorTok{$}\NormalTok{counts}
\NormalTok{tx_counts <-}\StringTok{ }\NormalTok{txi_trans}\OperatorTok{$}\NormalTok{counts}

\NormalTok{trans_TPM <-}\StringTok{ }\NormalTok{txi_trans}\OperatorTok{$}\NormalTok{abundance}
\end{Highlighting}
\end{Shaded}

\hypertarget{data-visualization}{%
\section{Data Visualization}\label{data-visualization}}

\hypertarget{httpsgithub.comwyguothreedrnaseqblobmastervignettesuser_manuals3d_rna-seq_command_line_user_manual.mdde-das-and-dtu-analysis}{%
\section{\texorpdfstring{\url{https://github.com/wyguo/ThreeDRNAseq/blob/master/vignettes/user_manuals/3D_RNA-seq_command_line_user_manual.md\#de-das-and-dtu-analysis}}{https://github.com/wyguo/ThreeDRNAseq/blob/master/vignettes/user\_manuals/3D\_RNA-seq\_command\_line\_user\_manual.md\#de-das-and-dtu-analysis}}\label{httpsgithub.comwyguothreedrnaseqblobmastervignettesuser_manuals3d_rna-seq_command_line_user_manual.mdde-das-and-dtu-analysis}}

\hypertarget{step-2-filter-low-expression-genes}{%
\section{Step 2: Filter low expression
genes}\label{step-2-filter-low-expression-genes}}

\#\#-----\textgreater{}\textgreater{} Do the filters

\begin{Shaded}
\begin{Highlighting}[]
\NormalTok{target_high <-}\StringTok{ }\KeywordTok{low.expression.filter}\NormalTok{(}\DataTypeTok{abundance =}\NormalTok{ tx_counts, }
                                     \DataTypeTok{mapping =}\NormalTok{ tx2gene,}
                                     \DataTypeTok{abundance.cut =} \DecValTok{1}\NormalTok{,}
                                     \DataTypeTok{sample.n =} \DecValTok{3}\NormalTok{,}
                                     \DataTypeTok{unit =} \StringTok{'counts'}\NormalTok{,}
                                     \DataTypeTok{Log=}\NormalTok{F)}
\end{Highlighting}
\end{Shaded}

\begin{verbatim}
## Warning: Setting row names on a tibble is deprecated.
\end{verbatim}

Mean-variance plot

\hypertarget{gene-level}{%
\subsection{-----\textgreater{}\textgreater{} gene
level}\label{gene-level}}

\begin{Shaded}
\begin{Highlighting}[]
\NormalTok{counts.raw =}\StringTok{ }\NormalTok{genes_counts[}\KeywordTok{rowSums}\NormalTok{(genes_counts}\OperatorTok{>}\DecValTok{0}\NormalTok{)}\OperatorTok{>}\DecValTok{0}\NormalTok{,]}

\NormalTok{counts.filtered =}\StringTok{ }\NormalTok{genes_counts[target_high}\OperatorTok{$}\NormalTok{genes_high,]}

\NormalTok{mv.genes <-}\StringTok{ }\KeywordTok{check.mean.variance}\NormalTok{(}\DataTypeTok{counts.raw =}\NormalTok{ counts.raw,}
                                \DataTypeTok{counts.filtered =}\NormalTok{ counts.filtered,}
                                \DataTypeTok{condition =}\NormalTok{ metadata}\OperatorTok{$}\NormalTok{treat)}
\end{Highlighting}
\end{Shaded}

\begin{verbatim}
## => Generate DGEList object
\end{verbatim}

\begin{verbatim}
## => Fit mean-variance trend
\end{verbatim}

\begin{verbatim}
## Done!!!
\end{verbatim}

\hypertarget{make-plot}{%
\subsubsection{make plot}\label{make-plot}}

\begin{Shaded}
\begin{Highlighting}[]
\NormalTok{fit.raw <-}\StringTok{ }\NormalTok{mv.genes}\OperatorTok{$}\NormalTok{fit.raw}
\NormalTok{fit.filtered <-}\StringTok{ }\NormalTok{mv.genes}\OperatorTok{$}\NormalTok{fit.filtered}
\NormalTok{mv.genes.plot <-}\StringTok{ }\ControlFlowTok{function}\NormalTok{()\{}
  \KeywordTok{par}\NormalTok{(}\DataTypeTok{mfrow=}\KeywordTok{c}\NormalTok{(}\DecValTok{1}\NormalTok{,}\DecValTok{2}\NormalTok{))}
  \KeywordTok{plotMeanVariance}\NormalTok{(}\DataTypeTok{x =}\NormalTok{ fit.raw}\OperatorTok{$}\NormalTok{sx,}\DataTypeTok{y =}\NormalTok{ fit.raw}\OperatorTok{$}\NormalTok{sy,}
                   \DataTypeTok{l =}\NormalTok{ fit.raw}\OperatorTok{$}\NormalTok{l,}\DataTypeTok{lwd=}\DecValTok{2}\NormalTok{,}\DataTypeTok{fit.line.col =}\StringTok{'gold'}\NormalTok{,}\DataTypeTok{col=}\StringTok{'black'}\NormalTok{)}
  \KeywordTok{title}\NormalTok{(}\StringTok{'}\CharTok{\textbackslash{}n\textbackslash{}n}\StringTok{Raw counts (gene level)'}\NormalTok{)}
  \KeywordTok{plotMeanVariance}\NormalTok{(}\DataTypeTok{x =}\NormalTok{ fit.filtered}\OperatorTok{$}\NormalTok{sx,}\DataTypeTok{y =}\NormalTok{ fit.filtered}\OperatorTok{$}\NormalTok{sy,}
                   \DataTypeTok{l =}\NormalTok{ fit.filtered}\OperatorTok{$}\NormalTok{l,}\DataTypeTok{lwd=}\DecValTok{2}\NormalTok{,}\DataTypeTok{col=}\StringTok{'black'}\NormalTok{)}
  \KeywordTok{title}\NormalTok{(}\StringTok{'}\CharTok{\textbackslash{}n\textbackslash{}n}\StringTok{Filtered counts (gene level)'}\NormalTok{)}
  \KeywordTok{lines}\NormalTok{(fit.raw}\OperatorTok{$}\NormalTok{l, }\DataTypeTok{col =} \StringTok{"gold"}\NormalTok{,}\DataTypeTok{lty=}\DecValTok{4}\NormalTok{,}\DataTypeTok{lwd=}\DecValTok{2}\NormalTok{)}
  \KeywordTok{legend}\NormalTok{(}\StringTok{'topright'}\NormalTok{,}\DataTypeTok{col =} \KeywordTok{c}\NormalTok{(}\StringTok{'red'}\NormalTok{,}\StringTok{'gold'}\NormalTok{),}\DataTypeTok{lty=}\KeywordTok{c}\NormalTok{(}\DecValTok{1}\NormalTok{,}\DecValTok{4}\NormalTok{),}\DataTypeTok{lwd=}\DecValTok{3}\NormalTok{,}
         \DataTypeTok{legend =} \KeywordTok{c}\NormalTok{(}\StringTok{'low-exp removed'}\NormalTok{,}\StringTok{'low-exp kept'}\NormalTok{))}
\NormalTok{\}}
\KeywordTok{mv.genes.plot}\NormalTok{()}
\end{Highlighting}
\end{Shaded}

\includegraphics{ANU_UWA_AD_files/figure-latex/unnamed-chunk-10-1.pdf}

Principal component analysis (PCA)

\#\#-----\textgreater{}\textgreater{} genes level

\begin{Shaded}
\begin{Highlighting}[]
\NormalTok{data2pca <-}\StringTok{ }\NormalTok{genes_counts[target_high}\OperatorTok{$}\NormalTok{genes_high,]}
\NormalTok{dge <-}\StringTok{ }\KeywordTok{DGEList}\NormalTok{(}\DataTypeTok{counts=}\NormalTok{data2pca) }
\NormalTok{dge <-}\StringTok{ }\KeywordTok{calcNormFactors}\NormalTok{(dge)}
\NormalTok{data2pca <-}\StringTok{ }\KeywordTok{t}\NormalTok{(}\KeywordTok{counts2CPM}\NormalTok{(}\DataTypeTok{obj =}\NormalTok{ dge,}\DataTypeTok{Log =}\NormalTok{ T))}
\NormalTok{dim1 <-}\StringTok{ 'PC1'}
\NormalTok{dim2 <-}\StringTok{ 'PC2'}
\NormalTok{ellipse.type <-}\StringTok{ 'polygon'} \CommentTok{#ellipse.type=c('none','ellipse','polygon')}

\CommentTok{##--All Bio-reps plots}

\NormalTok{groups <-}\StringTok{ }\NormalTok{metadata}\OperatorTok{$}\NormalTok{treat }\CommentTok{## colour on biological replicates}
\CommentTok{#groups <- metadata$label ## colour on condtions}
\NormalTok{g <-}\StringTok{ }\KeywordTok{plotPCAind}\NormalTok{(}\DataTypeTok{data2pca =}\NormalTok{ data2pca, }\DataTypeTok{dim1 =}\NormalTok{ dim1, }\DataTypeTok{dim2 =}\NormalTok{ dim2,}
                \DataTypeTok{groups =}\NormalTok{ groups, }\DataTypeTok{plot.title =} \StringTok{'genescript PCA: bio-reps'}\NormalTok{,}
                \DataTypeTok{ellipse.type =}\NormalTok{ ellipse.type,}
                \DataTypeTok{add.label =}\NormalTok{ F, }\DataTypeTok{adj.label =}\NormalTok{ F)}

\NormalTok{g}
\end{Highlighting}
\end{Shaded}

\includegraphics{ANU_UWA_AD_files/figure-latex/unnamed-chunk-11-1.pdf}

\#\#--average Gene expression plot

\begin{Shaded}
\begin{Highlighting}[]
\KeywordTok{rownames}\NormalTok{(data2pca) <-}\StringTok{ }\KeywordTok{gsub}\NormalTok{(}\StringTok{'_'}\NormalTok{,}\StringTok{'.'}\NormalTok{,}\KeywordTok{rownames}\NormalTok{(data2pca))}
\NormalTok{groups <-}\StringTok{ }\NormalTok{metadata}\OperatorTok{$}\NormalTok{label }\CommentTok{## colour on biological replicates}
\NormalTok{data2pca.ave <-}\StringTok{ }\KeywordTok{rowmean}\NormalTok{(data2pca,metadata}\OperatorTok{$}\NormalTok{treat,}\DataTypeTok{reorder =}\NormalTok{ F)}
\NormalTok{groups <-}\StringTok{ }\KeywordTok{unique}\NormalTok{(metadata}\OperatorTok{$}\NormalTok{treat)}
\NormalTok{g <-}\StringTok{ }\KeywordTok{plotPCAind}\NormalTok{(}\DataTypeTok{data2pca =}\NormalTok{ data2pca.ave,}\DataTypeTok{dim1 =} \StringTok{'PC1'}\NormalTok{,}\DataTypeTok{dim2 =} \StringTok{'PC2'}\NormalTok{,}
                \DataTypeTok{groups =}\NormalTok{ groups,}\DataTypeTok{plot.title =} \StringTok{'genescript PCA: average gene expression'}\NormalTok{,}
                \DataTypeTok{ellipse.type =} \StringTok{'none'}\NormalTok{,}\DataTypeTok{add.label =}\NormalTok{ T,}\DataTypeTok{adj.label =}\NormalTok{ F)}

\NormalTok{g}
\end{Highlighting}
\end{Shaded}

\includegraphics{ANU_UWA_AD_files/figure-latex/unnamed-chunk-12-1.pdf}

\#\#-----\textgreater{}\textgreater{} trans level

\begin{Shaded}
\begin{Highlighting}[]
\NormalTok{data2pca <-}\StringTok{ }\NormalTok{tx_counts[target_high}\OperatorTok{$}\NormalTok{trans_high,]}
\NormalTok{dge <-}\StringTok{ }\KeywordTok{DGEList}\NormalTok{(}\DataTypeTok{counts=}\NormalTok{data2pca) }
\NormalTok{dge <-}\StringTok{ }\KeywordTok{calcNormFactors}\NormalTok{(dge)}
\NormalTok{data2pca <-}\StringTok{ }\KeywordTok{t}\NormalTok{(}\KeywordTok{counts2CPM}\NormalTok{(}\DataTypeTok{obj =}\NormalTok{ dge,}\DataTypeTok{Log =}\NormalTok{ T))}
\NormalTok{dim1 <-}\StringTok{ 'PC1'}
\NormalTok{dim2 <-}\StringTok{ 'PC2'}
\NormalTok{ellipse.type <-}\StringTok{ 'polygon'} \CommentTok{#ellipse.type=c('none','ellipse','polygon')}

\CommentTok{##--All Bio-reps plots}
\NormalTok{groups <-}\StringTok{ }\NormalTok{metadata}\OperatorTok{$}\NormalTok{treat }\CommentTok{## colour on biological replicates}
\CommentTok{#groups <- metadata$label ## colour on condtions}
\NormalTok{g <-}\StringTok{ }\KeywordTok{plotPCAind}\NormalTok{(}\DataTypeTok{data2pca =}\NormalTok{ data2pca,}\DataTypeTok{dim1 =}\NormalTok{ dim1,}\DataTypeTok{dim2 =}\NormalTok{ dim2,}
                \DataTypeTok{groups =}\NormalTok{ groups,}\DataTypeTok{plot.title =} \StringTok{'Transcript PCA: bio-reps'}\NormalTok{,}
                \DataTypeTok{ellipse.type =}\NormalTok{ ellipse.type,}
                \DataTypeTok{add.label =}\NormalTok{ T,}\DataTypeTok{adj.label =}\NormalTok{ F)}

\NormalTok{g}
\end{Highlighting}
\end{Shaded}

\includegraphics{ANU_UWA_AD_files/figure-latex/unnamed-chunk-13-1.pdf}

\#\#--average transcript expression plot

\begin{Shaded}
\begin{Highlighting}[]
\NormalTok{groups <-}\StringTok{ }\NormalTok{metadata}\OperatorTok{$}\NormalTok{label}
\NormalTok{data2pca.ave <-}\StringTok{ }\KeywordTok{rowmean}\NormalTok{(data2pca,metadata}\OperatorTok{$}\NormalTok{treat,}\DataTypeTok{reorder =}\NormalTok{ F)}
\NormalTok{groups <-}\StringTok{ }\KeywordTok{unique}\NormalTok{(metadata}\OperatorTok{$}\NormalTok{treat)}
\NormalTok{g <-}\StringTok{ }\KeywordTok{plotPCAind}\NormalTok{(}\DataTypeTok{data2pca =}\NormalTok{ data2pca.ave,}\DataTypeTok{dim1 =} \StringTok{'PC1'}\NormalTok{,}\DataTypeTok{dim2 =} \StringTok{'PC2'}\NormalTok{,}
                \DataTypeTok{groups =}\NormalTok{ groups,}\DataTypeTok{plot.title =} \StringTok{'Transcript PCA: average expression'}\NormalTok{,}
                \DataTypeTok{ellipse.type =} \StringTok{'none'}\NormalTok{,}\DataTypeTok{add.label =}\NormalTok{ T,}\DataTypeTok{adj.label =}\NormalTok{ F)}

\NormalTok{g}
\end{Highlighting}
\end{Shaded}

\includegraphics{ANU_UWA_AD_files/figure-latex/unnamed-chunk-14-1.pdf}
\#\#\# save to figure

Data normalization

\hypertarget{data-normalisation-parameter}{%
\subsubsection{data normalisation
parameter}\label{data-normalisation-parameter}}

\begin{Shaded}
\begin{Highlighting}[]
\NormalTok{norm_method <-}\StringTok{ 'TMM'} \CommentTok{## norm_method is one of 'TMM','RLE' and 'upperquartile'}
\end{Highlighting}
\end{Shaded}

\#\#-----\textgreater{}\textgreater{} genes level

\begin{Shaded}
\begin{Highlighting}[]
\NormalTok{dge <-}\StringTok{ }\KeywordTok{DGEList}\NormalTok{(}\DataTypeTok{counts=}\NormalTok{genes_counts[target_high}\OperatorTok{$}\NormalTok{genes_high,],}
               \DataTypeTok{group =}\NormalTok{ metadata}\OperatorTok{$}\NormalTok{label)}
\NormalTok{genes_dge <-}\StringTok{ }\KeywordTok{suppressWarnings}\NormalTok{(}\KeywordTok{calcNormFactors}\NormalTok{(dge,}\DataTypeTok{method =}\NormalTok{ norm_method))}
\KeywordTok{save}\NormalTok{(genes_dge,}\DataTypeTok{file=}\KeywordTok{paste0}\NormalTok{(}\StringTok{'genes_dge.RData'}\NormalTok{))}
\end{Highlighting}
\end{Shaded}

\#\#-----\textgreater{}\textgreater{} MDS plot

\begin{Shaded}
\begin{Highlighting}[]
\NormalTok{group <-}\StringTok{  }\NormalTok{metadata}\OperatorTok{$}\NormalTok{label}

\NormalTok{y <-}\StringTok{ }\KeywordTok{calcNormFactors}\NormalTok{(dge)}
\NormalTok{design <-}\StringTok{ }\KeywordTok{model.matrix}\NormalTok{(}\OperatorTok{~}\NormalTok{group)}
\NormalTok{y <-}\StringTok{ }\KeywordTok{estimateDisp}\NormalTok{(y, design)}
\end{Highlighting}
\end{Shaded}

\begin{verbatim}
## Warning in estimateDisp.default(y = y$counts, design = design, group =
## group, : No residual df: setting dispersion to NA
\end{verbatim}

\begin{Shaded}
\begin{Highlighting}[]
\KeywordTok{par}\NormalTok{(}\DataTypeTok{mfrow=}\KeywordTok{c}\NormalTok{(}\DecValTok{1}\NormalTok{,}\DecValTok{1}\NormalTok{))}

\KeywordTok{plotMDS}\NormalTok{(y, }\DataTypeTok{labels=}\OtherTok{NULL}\NormalTok{, }\DataTypeTok{pch=} \DecValTok{19}\NormalTok{, }\DataTypeTok{cex=}\DecValTok{3}\NormalTok{, }\DataTypeTok{cex.axis=}\FloatTok{2.2}\NormalTok{, }\DataTypeTok{cex.lab=} \FloatTok{2.2}\NormalTok{, }\DataTypeTok{cex.main=}\DecValTok{3}\NormalTok{, }\DataTypeTok{col=} \KeywordTok{c}\NormalTok{(}\KeywordTok{rep}\NormalTok{(}\StringTok{"black"}\NormalTok{, }\DecValTok{3}\NormalTok{), }\KeywordTok{rep}\NormalTok{(}\StringTok{"bisque3"}\NormalTok{, }\DecValTok{3}\NormalTok{), }\KeywordTok{rep}\NormalTok{(}\StringTok{"bisque4"}\NormalTok{, }\DecValTok{2}\NormalTok{), }\KeywordTok{rep}\NormalTok{(}\StringTok{"#99CCFF"}\NormalTok{, }\DecValTok{3}\NormalTok{), }\KeywordTok{rep}\NormalTok{(}\StringTok{"#3399FF"}\NormalTok{, }\DecValTok{3}\NormalTok{), }\KeywordTok{rep}\NormalTok{(}\StringTok{"#0066CC"}\NormalTok{, }\DecValTok{3}\NormalTok{), }\KeywordTok{rep}\NormalTok{(}\StringTok{"#CCCC00"}\NormalTok{, }\DecValTok{3}\NormalTok{), }\KeywordTok{rep}\NormalTok{(}\StringTok{"#999900"}\NormalTok{, }\DecValTok{3}\NormalTok{), }\KeywordTok{rep}\NormalTok{(}\StringTok{"#666600"}\NormalTok{, }\DecValTok{3}\NormalTok{), }\KeywordTok{rep}\NormalTok{(}\StringTok{"#CC66FF"}\NormalTok{, }\DecValTok{3}\NormalTok{), }\KeywordTok{rep}\NormalTok{(}\StringTok{"#9900CC"}\NormalTok{, }\DecValTok{3}\NormalTok{), }\KeywordTok{rep}\NormalTok{(}\StringTok{"#660066"}\NormalTok{, }\DecValTok{3}\NormalTok{),}\KeywordTok{rep}\NormalTok{(}\StringTok{"#118833"}\NormalTok{, }\DecValTok{3}\NormalTok{),}\KeywordTok{rep}\NormalTok{(}\StringTok{"#118855"}\NormalTok{, }\DecValTok{3}\NormalTok{),}\KeywordTok{rep}\NormalTok{(}\StringTok{"#118899"}\NormalTok{, }\DecValTok{3}\NormalTok{)), }\DataTypeTok{main=} \StringTok{"Multidimentional scaling plot"}\NormalTok{)}
\end{Highlighting}
\end{Shaded}

\includegraphics{ANU_UWA_AD_files/figure-latex/unnamed-chunk-17-1.pdf}

\begin{Shaded}
\begin{Highlighting}[]
\NormalTok{text <-}\StringTok{ }\KeywordTok{c}\NormalTok{(}\StringTok{'atg2_0h'}\NormalTok{, }\StringTok{'atg2_12h'}\NormalTok{, }\StringTok{'atg2_24h'}\NormalTok{, }\StringTok{'atg5_0h'}\NormalTok{, }\StringTok{'atg5_12h'}\NormalTok{, }\StringTok{'atg5_24h'}\NormalTok{, }\StringTok{'atg7_0h'}\NormalTok{, }\StringTok{'atg7_12h'}\NormalTok{, }\StringTok{'atg7_24h'}\NormalTok{, }\StringTok{'atg9_0h'}\NormalTok{, }\StringTok{'atg9_12h'}\NormalTok{, }\StringTok{'atg9_24h'}\NormalTok{, }\StringTok{'WT_0h'}\NormalTok{, }\StringTok{'WT_12h'}\NormalTok{, }\StringTok{'WT_24h'}\NormalTok{)}
\KeywordTok{plot.new}\NormalTok{()}
\KeywordTok{par}\NormalTok{(}\DataTypeTok{xpd=}\OtherTok{TRUE}\NormalTok{)}
\NormalTok{plot_colors <-}\StringTok{ }\KeywordTok{c}\NormalTok{(}\StringTok{"black"}\NormalTok{,}\StringTok{"bisque3"}\NormalTok{, }\StringTok{"bisque4"}\NormalTok{, }\StringTok{"#99CCFF"}\NormalTok{, }\StringTok{"#3399FF"}\NormalTok{, }\StringTok{"#0066CC"}\NormalTok{, }\StringTok{"#CCCC00"}\NormalTok{, }\StringTok{"#999900"}\NormalTok{, }\StringTok{"#666600"}\NormalTok{, }\StringTok{"#CC66FF"}\NormalTok{, }\StringTok{"#9900CC"}\NormalTok{, }\StringTok{"#660066"}\NormalTok{, }\StringTok{"#118833"}\NormalTok{, }\StringTok{"#118855"}\NormalTok{, }\StringTok{"#118899"}\NormalTok{)}
\KeywordTok{legend}\NormalTok{(}\StringTok{"center"}\NormalTok{,}\DataTypeTok{legend =}\NormalTok{ text,}
       \DataTypeTok{col=}\NormalTok{plot_colors, }\DataTypeTok{cex=}\DecValTok{2}\NormalTok{, }\DataTypeTok{y.intersp =} \FloatTok{0.35}\NormalTok{, }\DataTypeTok{horiz =} \OtherTok{FALSE}\NormalTok{, }\DataTypeTok{pch=}\DecValTok{19}\NormalTok{, }\DataTypeTok{bty=}\StringTok{"n"}\NormalTok{)}
\end{Highlighting}
\end{Shaded}

\includegraphics{ANU_UWA_AD_files/figure-latex/unnamed-chunk-17-2.pdf}
\#\#\#\#\#\#\#\#\#\#\#\#\#\#\#\#\#\#\#\#\#\#\#\#\#\#\#\#\#\#\#\#\#\#\#\#\#\#\#\#\#\#\#\#\#\#\#\#\#\#\#\#\#\#\#\#\#\#\#\#\#\#\#\#\#\#\#\#\#\#\#\#\#\#\#\#\#\#\#\#
\#\#-----\textgreater{}\textgreater{} distribution plot

\begin{Shaded}
\begin{Highlighting}[]
\NormalTok{sample.name <-}\StringTok{ }\KeywordTok{paste0}\NormalTok{(metadata}\OperatorTok{$}\NormalTok{label,}\StringTok{'.'}\NormalTok{,metadata}\OperatorTok{$}\NormalTok{treat)}
\NormalTok{condition <-}\StringTok{ }\NormalTok{metadata}\OperatorTok{$}\NormalTok{label}
\end{Highlighting}
\end{Shaded}

\#\#\#--- genes level

\begin{Shaded}
\begin{Highlighting}[]
\NormalTok{data.before <-}\StringTok{ }\NormalTok{genes_counts[target_high}\OperatorTok{$}\NormalTok{genes_high,]}
\NormalTok{data.after <-}\StringTok{ }\KeywordTok{counts2CPM}\NormalTok{(}\DataTypeTok{obj =}\NormalTok{ genes_dge,}\DataTypeTok{Log =}\NormalTok{ T)}
\NormalTok{g <-}\StringTok{ }\KeywordTok{boxplotNormalised}\NormalTok{(}\DataTypeTok{data.before =}\NormalTok{ data.before,}
                       \DataTypeTok{data.after =}\NormalTok{ data.after,}
                       \DataTypeTok{condition =}\NormalTok{ condition,}
                       \DataTypeTok{sample.name =}\NormalTok{ sample.name)}
\KeywordTok{do.call}\NormalTok{(grid.arrange,g)}
\end{Highlighting}
\end{Shaded}

\includegraphics{ANU_UWA_AD_files/figure-latex/unnamed-chunk-19-1.pdf}

\#\#-----\textgreater{}\textgreater{} trans level

\begin{Shaded}
\begin{Highlighting}[]
\NormalTok{dge <-}\StringTok{ }\KeywordTok{DGEList}\NormalTok{(}\DataTypeTok{counts=}\NormalTok{tx_counts[target_high}\OperatorTok{$}\NormalTok{trans_high,],}
               \DataTypeTok{group =}\NormalTok{ metadata}\OperatorTok{$}\NormalTok{label,}
               \DataTypeTok{genes =}\NormalTok{ tx2gene[target_high}\OperatorTok{$}\NormalTok{trans_high,])}
\NormalTok{trans_dge <-}\StringTok{ }\KeywordTok{suppressWarnings}\NormalTok{(}\KeywordTok{calcNormFactors}\NormalTok{(dge,}\DataTypeTok{method =}\NormalTok{ norm_method))}
\KeywordTok{save}\NormalTok{(trans_dge,}\DataTypeTok{file=}\KeywordTok{paste0}\NormalTok{(}\StringTok{'trans_dge.RData'}\NormalTok{))}


\CommentTok{################################################################################}
\CommentTok{##----->> distribution plot}
\NormalTok{sample.name <-}\StringTok{ }\KeywordTok{paste0}\NormalTok{(metadata}\OperatorTok{$}\NormalTok{label,}\StringTok{'.'}\NormalTok{,metadata}\OperatorTok{$}\NormalTok{treat)}
\NormalTok{condition <-}\StringTok{ }\NormalTok{metadata}\OperatorTok{$}\NormalTok{label}

\CommentTok{###--- trans level}
\NormalTok{data.before <-}\StringTok{ }\NormalTok{tx_counts[target_high}\OperatorTok{$}\NormalTok{trans_high,]}
\NormalTok{data.after <-}\StringTok{ }\KeywordTok{counts2CPM}\NormalTok{(}\DataTypeTok{obj =}\NormalTok{ trans_dge,}\DataTypeTok{Log =}\NormalTok{ T)}
\NormalTok{g <-}\StringTok{ }\KeywordTok{boxplotNormalised}\NormalTok{(}\DataTypeTok{data.before =}\NormalTok{ data.before,}
                       \DataTypeTok{data.after =}\NormalTok{ data.after,}
                       \DataTypeTok{condition =}\NormalTok{ condition,}
                       \DataTypeTok{sample.name =}\NormalTok{ sample.name)}
\KeywordTok{do.call}\NormalTok{(grid.arrange,g)}
\end{Highlighting}
\end{Shaded}

\includegraphics{ANU_UWA_AD_files/figure-latex/unnamed-chunk-20-1.pdf}


\end{document}
